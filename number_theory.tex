% number_theory.tex
% 2017/07/20, v1

\chapter{数论}
\label{number_theory}

\newcommand{\prefixnumbthy}[1]{number\_theory/#1}
\newcommand{\xprefixnumbthy}[1]{number_theory/#1}


\begin{summary}
本章算法主要关于整除、素数、最大公约数和同余等问题。重点内容有:求最大公约数的欧几里德算法、线性欧拉筛、扩展欧几里德算法、同余方程、中国剩余定理、快速幂、逆元、欧拉函数。
\end{summary}

\xsection{欧几里德gcd算法}{\prefixnumbthy{gcd.cpp}}

\lstinputlisting{\xprefixnumbthy{gcd.cpp}}

\xsection{扩展欧几里德算法}{\prefixnumbthy{extend\_gcd.cpp}}

\lstinputlisting{\xprefixnumbthy{extend_gcd.cpp}}


\section{素数相关算法}

\xsubsection{判断$n$是否素数}{\prefixnumbthy{is\_prime.cpp}}
\lstinputlisting{\xprefixnumbthy{is_prime.cpp}}


\xsubsection{线性筛(欧拉筛)求素数表}{\prefixnumbthy{eular\_prime.cpp}}
\begin{example}[素数]
\href{https://xoj.red/problems/show/1923}{XOJ 1923 素数}
\end{example}
\lstinputlisting{\xprefixnumbthy{eular_prime.cpp}}


\xsubsection{质因数分解}{\prefixnumbthy{factors.cpp}}
\begin{example}[质因数分解]
\href{https://xoj.red/problems/show/1626}{XOJ 1626 质因数分解}
\end{example}
\lstinputlisting{\xprefixnumbthy{factors.cpp}}




\section{欧拉函数$\varphi(x)$}
$\varphi(n)$表示小于$n$并与$n$互质的正整数的数目,计算公式如下:
\[
\varphi(n) = n \prod_{i=1}^n (1-\frac{1}{p_i})
\]

其中$p_i \; (i \in [1,n])$是$n$的所有质因子,即$n$的质因数分解是:
\[
n = p_1^{a_1} \cdot p_2^{a_2} \cdot p_3^{a_3} \cdots p_k^{a_k}
\]
\begin{example}[素数]
\href{https://xoj.red/problems/show/3329}{XOJ 3329 Relatives}
\end{example}
\xsubsection{分解质因数法}{\prefixnumbthy{phi\_factors.cpp}}
\lstinputlisting{\xprefixnumbthy{phi_factors.cpp}}

\xsubsection{分解质因数法(单独求解)}{\prefixnumbthy{phi\_decompose.cpp}}
\lstinputlisting{\xprefixnumbthy{phi_decompose.cpp}}

\xsubsection{筛法欧拉函数}{\prefixnumbthy{phi\_filter.cpp}}
\lstinputlisting{\xprefixnumbthy{phi_filter.cpp}}

\xsubsection{线性筛~欧拉函数和素数表}{\prefixnumbthy{phi\_euler.cpp}}
\lstinputlisting{\xprefixnumbthy{phi_euler.cpp}}


\section{快速幂}

\xsubsection{二分快速幂取模}{\prefixnumbthy{power\_mod.cpp}}
\lstinputlisting{\xprefixnumbthy{power_mod.cpp}}


\subsection{二分快速幂(一般高精度)}


\section{逆元}


\section{同余方程(组)}

\xsubsection{同余方程}{\prefixnumbthy{mod\_equation.cpp}}
\begin{example}[同余方程]
\href{https://xoj.red/problems/show/1637}{XOJ 1637 同余方程}
\end{example}
\lstinputlisting{\xprefixnumbthy{mod_equation.cpp}}

\xsubsection{同余方程组(中国剩余定理)}{\prefixnumbthy{mod\_equation2.cpp}}
\begin{example}[中国剩余定理]
\href{https://xoj.red/problems/show/3335}{XOJ 3335 生理周期}
\end{example}
\lstinputlisting{\xprefixnumbthy{mod_equation2.cpp}}

\xsubsection{同余方程组(不互质)}{\prefixnumbthy{mod\_equation3.cpp}}
\begin{example}[同余方程组]
\href{https://xoj.red/problems/show/3336}{XOJ 3336 表示整数的奇怪方法}
\end{example}
\lstinputlisting{\xprefixnumbthy{mod_equation3.cpp}}




\section{高精度}
\subsection{高精度类}
\lstinputlisting{number_theory/big_integer.cpp}


\endinput % number_thery
