% tactics.tex
% 2017/07/20, v1

\chapter{思考策略及优化}
\label{tactics}


\section{二分查找}
\subsection{查找单值$v$}
\lstinputlisting{tactics/binary1.cpp}

\subsection{查找大于等于v的第一个值}
\lstinputlisting{tactics/binary2.cpp}

\subsection{查找小于等于v的最后一个值}
\lstinputlisting{tactics/binary3.cpp}

\subsection{二分套二分}
\lstinputlisting{tactics/binary4.cpp}


\section{输入 / 输出外挂}

我们都知道,scanf()、printf()、cin、cout其实就是对其他一些基础的获取或输出语句(getchar() putchar()等)进行封装,而这些基础的函数功能弱,效率高,所以我们的输入输出外挂也是仿照着scanf()、printf()、cin、cout来实现的,只不过做了针对性的改造,最终我们改造出来多种功能比scanf()等弱、比getchar()等强,效率比scanf()等高、比gerchar()等低的函数,从而达到针对性的作用,减少了不必要的资源消耗。

当然输入输出外挂一般用在大量输入输出的情况下,这样性价比才高一些,否则得不偿失(牺牲了代码长度而换来了微不足道的效率提升)。

一般情况下,加\verb|std::ios::sync_with_stdio(false);|取消输入输出同步即可,但注意不要C/C++混用!

\subsection{适用于正整数}
\lstinputlisting{tactics/input_output1.cpp}


\subsection{正负整数}
\lstinputlisting{tactics/input_output2.cpp}




\endinput % tactics
